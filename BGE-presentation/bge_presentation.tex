%!TEX program = xelatex
\documentclass{beamer}

\def\image[#1][#2]{
	\begin{figure}[H]
		\centering
		\includegraphics[#2]{#1}
\end{figure}}

\def\captionimage[#1][#2]#3{
	\begin{figure}[H]
		\centering
		\includegraphics[#2]{#1}
		\caption{#3}
\end{figure}}

\logo{\includegraphics[height=0.8cm]{logo.png}\vspace{239pt}}


\usetheme{mhthm}

\title{Blender Game Engine}
\subtitle{Università degli Studi di Perugia}
\author{Arianna Masciolini e Giorgio Mazza}
\institute{UniPG}


\date{Maggio 2018}

\setcounter{showSlideNumbers}{1}

\begin{document}
	\setcounter{showProgressBar}{0}
	\setcounter{showSlideNumbers}{0}

	\frame{\titlepage}
	\begin{frame}
		\frametitle{Sommario}
		\begin{itemize}
			\item Introduzione \\ {\footnotesize\hspace{1em} Casi d'uso e caratteristiche generali per il Blender Game Engine}
			\item Strumenti \\ {\footnotesize\hspace{1em} Editor grafico e scripting con Python}
			\item Componenti \\ {\footnotesize\hspace{1em} Sensori, controller e attuatori}
		\end{itemize}
	\end{frame}

	\setcounter{framenumber}{0}
	\setcounter{showProgressBar}{1}
	\setcounter{showSlideNumbers}{1}
	\section{Introduzione}
		\begin{frame}
			\frametitle{Caratteristiche generali}
			\begin{itemize}
				\item Casi d'uso: sviluppo di giochi e simulazioni interattive in 3D 
				\item C++
				\item Rendering in tempo reale
				\item Interfaccia: editor logico con interfaccia grafica 
				\item Supportato scripting in Python
				\item Possibilità di esportare l'applicazione per ambienti Linux, macOS, e MS-Windows tramite il plugin \textit{Save as Runtime}
			\end{itemize}
		\end{frame}
		\begin{frame}
			\frametitle{Licenza}
			Il BGE, in quanto parte di Blender, è rilasciato sotto la licenza \hyperlink{https://www.gnu.org/licenses/gpl-3.0.en.html}{\textcolor{BlenderOrange}{GNU GPL}}. Di conseguenza, i giochi e le simulazioni che contengono software Blender devono essere sottoposti ad una licenza compatibile (non così i modelli, le texture, i suoni etc.).
			Quando il gioco viene salvato come un unico file \textit{standalone}, il blend-file viene accorpato al software (il BGE player), per cui dovrà essere anch'esso compatibile con la GNU GPL. In ogni caso, è possibile caricare una blend-file autonomo dal player, cui può essere applicata qualsiasi licenza.
			\image[images/gpl.png][scale=0.4]
		\end{frame}
		\begin{frame}
			\frametitle{Librerie}
			\begin{itemize}
				\item Audaspace \\ {\footnotesize\hspace{1em} Controllo dell'audio}
				\item Bullet \\ {\footnotesize\hspace{1em} Rilevamento delle collisioni e simulazione di fisica} 
				\item Recast \\ {\footnotesize\hspace{1em} Costruzione di percorsi}
				\item Detour \\ {\footnotesize\hspace{1em} Pathfinding e ragionamento spaziale}
			\end{itemize}
		\end{frame}
		\begin{frame}
			\frametitle{Prospettive}
			\begin{quotation}\small 
				"(...) I also propose to refocus the current game engine (...) I propose to make the GE to become a real part of Blender code – to make it not separated anymore. This would make it more supported, more stable and (I”m sure) much more fun to work on as well. (...) What should then be dropped is the idea to make Blender have an embedded “true” game engine. (...) We never managed to make something with the portability and quality of Unreal or Crysis… or even Unity3D. And Blender's GPL license is not helping here much either. (...) The main cool feature of our GE is that it was integrated with a 3D tool, to allow people to make 3D interaction for walkthroughs, for scientific sims, or game prototypes. If we bring back this (original) design focus for a GE, I think we still get something unique and cool, with seamless integration of realtime and “offline” 3D". 
				\\ \textcolor{BlenderOrange}{\textbf{-Ton Roosendaal, creator of Blender}, \href{https://code.blender.org/2013/06/blender-roadmap-2-7-2-8-and-beyond/}{Blender roadmap}} 
			\end{quotation} 
		\end{frame}
		\begin{frame}
			\frametitle{Fasi dello sviluppo}
			\begin{enumerate}
				\item Creazione degli elementi visuali (modelli 3d o immagini)
				\item Uso dei blocchi logici per permettere l'interazione con la scena (programmazione a eventi)
				\item Creazione di una o più "macchine da presa" per il rendering e modifica dei parametri di visualizzazione
				\item Esportazione dell'applicazione
			\end{enumerate}
		\end{frame}
	
	\section{Strumenti}
		\begin{frame}
			\frametitle{Editor grafico}
			\begin{itemize}
				\item L'editor Logico è il metodo principale di impostare e modificare la logica del gioco per i vari attori che lo compongono
				\item La logica dei singoli oggetti selezionati nella 3D View è rappresentata dai Blocchi Logici (\textbf{Logic Bricks})
			\end{itemize}
		\end{frame}
		\begin{frame}
			\frametitle{Editor grafico - game logic}
			\image[images/gamegui.png][scale=0.16]
			Interfaccia consigliata per l'uso del BGE
		\end{frame}
		\begin{frame}
			\frametitle{Editor Grafico - logic bricks}
			\image[images/logic_bricks.png][scale=0.27]
			Funzioni programmate. Sono modificabili, combinabili e:
				\begin{itemize}
					\item sono suddivisi in tre tipi (colonne): 
					\begin{itemize}
						\item\textbf{Sensors}: \footnotesize si attivano al verificarsi di un evento, 
						\item \textbf{Controllers}: \footnotesize operano sul'input che ricevono dai sensori
						\item \textbf{Actuators}: \footnotesize interagiscono con il gioco.
					\end{itemize}
						Se un Sensore è positivo invia un impulso al Controllore che valuta la condizione ed innesca l’Attuatore.
						\item sono collegati tra loro mediante \textbf{link}, che conducono il flusso logico tra \textit{sensor e controller} e \textit{controller e actuator}
					\end{itemize}
		\end{frame}
		\begin{frame}
			\frametitle{Editor Grafico - properties}
			\image[images/props.png][scale=0.4]
			\textbf{Properties}: variabili dei linguaggi di programmazione; Usate per salvare ed accedere dati associati ad un oggetto o del gioco stesso
		\end{frame}
		\begin{frame}
			\frametitle{Editor Grafico - states}
			\image[images/states.png][scale=0.3]
			\textbf{States}: stati in cui può trovarsi un oggetto; usati per definire comportamenti diversi in situazioni diverse %sleeping, awake, dead
		\end{frame}
		\begin{frame}
		\frametitle{Scripting con Python - 1}
		Motivazioni per usare Python nel BGE:
			\begin{itemize}
				\item possibilità di gestire più oggetti contemporaneamente
				\item accesso a funzionalità avanzate di Blender
				\item accesso a funzionalità del Python non presenti in Blender
				\item possibilità di controllare dispositivi esterni
				\item accesso a strumenti di debugging.
			\end{itemize}
		\end{frame}	
		\begin{frame}
		\frametitle{Scripting con Python - 2}
		Va comunque tenuto presente che \textcolor{BlenderOrange}{il Python non può essere utilizzato \textit{al posto} dei blocchi logici}: nel BGE, uno script può essere visto come un controller più sofisticato, da utilizzare al posto di quelli booleani, che determinerà la relazione che intercorre tra sensori e attuatori di un oggetto. Essi, infatti -i sensori, 	gli attuatori e persino l'oggetto da cui è chiamato lo script- sono tutti attributi del controller, e in ogni caso gli script devono essere necessariamente chiamati da un controller.
		\end{frame}	
		\begin{frame}
			\frametitle{Scripting con Python - esempio}
		\end{frame}
	
	\section{Componenti}
		\begin{frame}		%SENSORS
			\frametitle{Logic - Sensori}
			\begin{itemize}
				\item \textit{(Event listeners)}: \textcolor{BlenderOrange}{cambiano stato} quando l'evento che attendono si verifica.
				\item Di default, i sensori \textcolor{BlenderOrange}{innescano i Controller} ad essi collegati ad ogni cambio di stato:
				\begin{itemize}
					\item \textit{"True level triggering"} \\ {\footnotesize\hspace{1em} innesta i controllers solo se lo stato del sensore è positivo}
					\item \textit{"False level triggering"} \\ {\footnotesize\hspace{1em}innesta i controllers solo se lo stato del sensore è negativo}
					\item \textit{"Skip"} \\ {\footnotesize\hspace{1em}imposta il ritardo tra i trigger ripetuti, in frame (tick logici)}
				\end{itemize}
			\end{itemize}
		%? img
		\end{frame}		
		\begin{frame}
			\frametitle{Logic - Tipi di Sensori}
			I principali sensori, aggiungibili con il pulsante "Add Sensor":\\
			\begin{quote}
			\textcolor{BlenderOrange}{\textbf{Nota}: Un sensore invia un impulso positivo (TRUE) a tutti i suoi Controllers quando si attiva; quando si disattiva invia un impulso negativo (FALSE)}
			\end{quote}
						
			\begin{itemize}
				\item \textbf{Actuator}: rileva quando un particolare Actuator riceve un impulso di attivazione o di disattivazione
				\item \textbf{Always}: innesca sempre i relativi Controllers (ad ogni tick logico)
				\item \textbf{Collision}: rileva quando l'oggetto proprietario collide (tocca) un'altro oggetto
				\item \textbf{Delay}: ritarda l'invio dell'impulso positivo
			\end{itemize}
		\end{frame}	
		\begin{frame}
			\frametitle{Logic - Tipi di Sensori - 2}
			I principali sensori, aggiungibili con il pulsante "Add Sensor":
			\begin{quote}
			\textcolor{BlenderOrange}{\textbf{Nota}: Un sensore invia un impulso positivo (TRUE) a tutti i suoi Controllers quando si attiva; quando si disattiva invia un impulso negativo (FALSE)}
			\end{quote}
			\begin{itemize}
				\item \textbf{Keyboard}: rileva la pressione di un tasto da noi assegnato %Joystick--analogo
				\item \textbf{Mouse}: rileva gli eventi del mouse. \\ {\footnotesize\hspace{1em}Eventi: \textit{Over any, Over, Movement, Wheel down/up, R/L/M button.}}
				\item \textbf{Near}: si attiva quando rileva degli oggetti ad una distanza x dall'oggetto proprietario.
				\item \textbf{Properties}: rileva cambiamenti nelle Proprietà (variabili) del suo oggetto.
			\end{itemize}
		\end{frame}	
		
		\begin{frame}
			\frametitle{Logic - Controllers} %CONTROLLERS

			Blocchi logici che \textcolor{BlenderOrange}{eseguono operazioni} sull'output del Sensore ed, in base a determinate condizioni, \textcolor{BlenderOrange}{attivano gli Actuators} ad esso connessi.
			\begin{enumerate}
				\item \textbf{States}: mostra la griglia degli stati, per spostare il Controller (e tutti i suoi blocchi collegati) in un altro State dell'oggetto.
				\item Esegue il Controller prima degli altri
			\end{enumerate}
			\image[images/contr.png][scale=0.3]
		\end{frame}	
		\begin{frame}
			\frametitle{Logic - Tipi di Controllers}
			I principali controllori, aggiungibili con il pulsante "Add Controller":
			\begin{quote}
			\textcolor{BlenderOrange}{\textbf{Nota}: un impulso positivo (TRUE) inviato dal Controller, attiva i suoi Actuators; uno negativo (FALSE) li disattiva.}
			\end{quote}
			\begin{itemize}
				\item \textbf{And}: {\footnotesize TRUE se riceve \textbf{tutti} impulsi positivi, FALSE altrimenti;}
				\item \textbf{Or}: {\footnotesize TRUE se riceve \textbf{almeno} un impulso positivo, FALSE altrimenti;}
				\item \textbf{Nand}: {\footnotesize TRUE se riceve \textbf{almeno} un impulso negativo, FALSE se riceve \textbf{tutti} impulsi positivi;}
				\item \textbf{Nor}: {\footnotesize TRUE se non riceve \textbf{nessun} impulso positivo, FALSE altrimenti;}
				\item \textbf{Xor}: {\footnotesize TRUE se riceve \textbf{un solo} impulso positivo, FALSE altrimenti;}
				\item \textbf{Xnor}: {\footnotesize TRUE se riceve \textbf{un solo} impulso negativo, FALSE altrimenti;}
				\item \textbf{Expression}: {\footnotesize valuta un'espressione booleana scritta dall'utente;}
				\item \textbf{Python}: {\footnotesize scripting in python, v. slides successive.}
			\end{itemize}
		\end{frame}
		
		
		\begin{frame}
			\frametitle{Logic - Actuators} %ACTUATORS
			\begin{itemize}
				\item Blocchi logici che interagiscono direttamente con il gioco
				\item Eseguono le loro funzioni solo quando ricevono un \textbf{impulso positivo} dai Controllers ad essi collegati
			\end{itemize}
		\end{frame}			
		\begin{frame}
			\frametitle{Logic - Tipi di Actuators}
			I principali attuatori, aggiungibili con il pulsante "Add Actuators":
			\begin{itemize}
				\item \textbf{Action}: {\footnotesize Riproduce e controlla le azioni di animazioni (visibile solo se è selezionata un'armatura);} %le azioni sono salvate nell'armatura!
				\item \textbf{Camera}: {\footnotesize fa sì che la camera segua un determinato oggetto;}
				\item \textbf{Constraints}: {\footnotesize  pone un vincolo sull’oggetto proprietario; \\I possibili vincoli sono: Orientation Distance Location}
				\item \textbf{Edit Object}: {\footnotesize permette di modificare l'oggetto proprietario; \\I possibili moidificatori sono: track to, replace mesh, end/add object;}
				\item \textbf{Filter 2D}: {\footnotesize aggiunge filtri applicati al render finale degli oggetti;}
				\item \textbf{Game}: {\footnotesize controlla l'intero gioco (es. ricomincia, carica, esci..);}
				\item \textbf{Mouse}: {\footnotesize Permette la visibilità del cursore e la rotazione tramite il mouse;}
			\end{itemize}
		\end{frame}
				\begin{frame}
			\frametitle{Logic - Tipi di Actuators - 2}
			I principali attuatori, aggiungibili con il pulsante "Add Actuators":
			\begin{itemize}
				\item \textbf{Motion}: {\footnotesize controlla il movimento di un oggetto; I possibili tipi di movimento sono: Simple, Servo control, Character.}
				\item \textbf{Parent}
				\item \textbf{Property}
				\item \textbf{Scene}: {\footnotesize controlla le scene nel \textit{blend-file};}
				\item \textbf{Filter 2D}: {\footnotesize aggiunge filtri applicati al render finale degli oggetti;}
				\item \textbf{Steering}: {\footnotesize fornisce vari metodi di pathfinding per un determinato oggetto;\\I vari metodi possono essere: Seek, Flee, Path following;}
				\item \textbf{Sound}
				\item \textbf{State}: {\footnotesize permette di gestire i vari stati del gioco;}
			\end{itemize}
		\end{frame}
		
		\begin{frame}
			\frametitle{Logic - Properties}
			\begin{itemize}
				\item Equivalenti alle variabili nei linguaggi di programmazione
				\item Memorizzate nell’oggetto proprietario
				\item \textbf{Tipi di proprietà:}: Timer, Float, Integer, String, Boolean
				\item Visualizzate a schermo spesso per motivi di Debug (i)
			\end{itemize}
		\end{frame}	
		\begin{frame}
			\frametitle{Logic - States}
			\begin{itemize}
				\item Un oggetto può possedere stati diversi, ad ogni stato corrisponde un comportamento diverso %ex expl->sleep/awake/dead
				\item Sono impostati ed usati mediante i \textit{Controllers}: solo questi ultimi sono direttamente controllati dal sistema di stati! \image[images/states.png][scale=0.3]
				\item Ogni oggetto può trovarsi in un solo stato ad un determinato istante di tempo
			\end{itemize}
		\end{frame}	
		\begin{frame}
			\frametitle{Logic - States - 2}
			\begin{itemize}
				\item\textbf{Visible}: sono tutti gli stati disponibili, quando uno state contiene dei logic bricks, in esso compare un pallino. Gli states visibili sono rappresentati dai quadrati di colore differente.
				\item\textbf{Initial}: determina in quale tra gli states l'oggetto si troverà appena avviato il gioco
			\end{itemize}
		\end{frame}	
\end{document}
