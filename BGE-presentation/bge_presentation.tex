%!TEX program = xelatex
\documentclass{beamer}

\def\code#1{\texttt{#1}}

\def\image[#1][#2]{
	\begin{figure}[H]
		\centering
		\includegraphics[#2]{#1}
\end{figure}}

\def\captionimage[#1][#2]#3{
	\begin{figure}[H]
		\centering
		\includegraphics[#2]{#1}
		\caption{#3}
\end{figure}}

\logo{\includegraphics[height=0.8cm]{logo.png}\vspace{239pt}}


\usetheme{mhthm}

\title{Blender Game Engine}
\subtitle{Università degli Studi di Perugia}
\author{Arianna Masciolini e Giorgio Mazza}
\institute{UniPG}


\date{Maggio 2018}

\setcounter{showSlideNumbers}{1}

\begin{document}
	\setcounter{showProgressBar}{0}
	\setcounter{showSlideNumbers}{0}

	\frame{\titlepage}
	\begin{frame}
		\frametitle{Sommario}
		\begin{itemize}
			\item Introduzione \\ {\footnotesize\hspace{1em} Casi d'uso, caratteristiche generali, problematiche e licenza}
			\item Interfaccia \\ {\footnotesize\hspace{1em} Editor grafico, editor di testo e console Python}
			\item Componenti \\ {\footnotesize\hspace{1em} Sensori, controller e attuatori}
		\end{itemize}
	\end{frame}

	\setcounter{framenumber}{0}
	\setcounter{showProgressBar}{1}
	\setcounter{showSlideNumbers}{1}
	\section{Introduzione}
		\begin{frame}
			\frametitle{Caratteristiche generali}
			\begin{itemize}
				\item Casi d'uso: sviluppo di giochi e simulazioni interattive in 3D 
				\item C++
				\item Rendering in tempo reale
				\item Interfaccia: editor logico con interfaccia grafica 
				\item Supportato scripting in Python
				\item Possibilità di esportare l'applicazione per ambienti Linux, macOS, e MS-Windows tramite il plugin \textit{Save as Runtime}
			\end{itemize}
		\end{frame}
		\begin{frame}
			\frametitle{Licenza}
			Il BGE, in quanto parte di Blender, è rilasciato sotto la licenza \hyperlink{https://www.gnu.org/licenses/gpl-3.0.en.html}{\textcolor{BlenderOrange}{GNU GPL}}. Di conseguenza, i giochi e le simulazioni che contengono software Blender devono essere sottoposti ad una licenza compatibile (non così i modelli, le texture, i suoni etc.).
			Quando il gioco viene salvato come unico file \textit{standalone}, il blend-file viene accorpato al BGE player), per cui dovrà essere anch'esso compatibile con la GNU GPL. In ogni caso, è possibile utilizzare un blend-file autonomo dal player, cui può essere applicata qualsiasi licenza.
			\image[images/gpl.png][scale=0.4]
		\end{frame}
		\begin{frame}
			\frametitle{Librerie}
			\begin{itemize}
				\item Audaspace \\ {\footnotesize\hspace{1em} Controllo dell'audio}
				\item Bullet \\ {\footnotesize\hspace{1em} Rilevamento delle collisioni e simulazione di fisica} 
				\item Recast \\ {\footnotesize\hspace{1em} Costruzione di percorsi}
				\item Detour \\ {\footnotesize\hspace{1em} Pathfinding e ragionamento spaziale}
			\end{itemize}
		\end{frame}
		\begin{frame}
			\frametitle{Problematiche}
			\begin{itemize}
				\item Benché il BGE sia integrato in Blender, il codice del motore di gioco è separato dal resto, per cui:
				\begin{itemize}
					\item Non tutte le funzionalità di Blender sono disponibili nel motore di gioco
					\item Ogni bug fix in Blender deve essere replicato nell'engine
				\end{itemize}
				\item Documentazione incompleta, specialmente per quel che riguarda l'uso di Python
				\item Gli stessi sviluppatori sconsigliano di avviare progetti a lungo termine in Blender in quanto è in programma una totale riscrittura dell'engine, che renderà la nuova versione incompatibile con le precedenti
			\end{itemize}
		\end{frame}
		\begin{frame}
			\frametitle{Prospettive}
			\begin{quotation}\small 
				"(...) I also propose to refocus the current game engine (...) I propose to make the GE to become a real part of Blender code – to make it not separated anymore. This would make it more supported, more stable and (I”m sure) much more fun to work on as well. (...) What should then be dropped is the idea to make Blender have an embedded “true” game engine. (...) We never managed to make something with the portability and quality of Unreal or Crysis… or even Unity3D. And Blender's GPL license is not helping here much either. (...) The main cool feature of our GE is that it was integrated with a 3D tool, to allow people to make 3D interaction for walkthroughs, for scientific sims, or game prototypes. If we bring back this (original) design focus for a GE, I think we still get something unique and cool, with seamless integration of realtime and “offline” 3D". 
				\\ \textcolor{BlenderOrange}{\textbf{-Ton Roosendaal, creator of Blender}, \href{https://code.blender.org/2013/06/blender-roadmap-2-7-2-8-and-beyond/}{Blender roadmap}} 
			\end{quotation} 
		\end{frame}
		\begin{frame}
			\frametitle{Fasi dello sviluppo}
			\begin{enumerate}
				\item Creazione degli elementi visuali (modelli 3d o immagini)
				\item Uso dei blocchi logici per permettere l'interazione con la scena (programmazione a eventi)
				\item Creazione di una o più "macchine da presa" per il rendering e modifica dei parametri di visualizzazione
				\item Esportazione dell'applicazione
			\end{enumerate}
		\end{frame}
	
	\section{Interfaccia}
		\begin{frame}
			\frametitle{Game logic}
			\image[images/gamegui.png][scale=0.16]
			Interfaccia consigliata per l'uso del BGE
		\end{frame}
		\begin{frame}
			\frametitle{Editor logico}
			\begin{itemize}
				\item L'editor logico è il metodo principale di impostare e modificare la logica del gioco per i vari attori che lo compongono
				\item La logica dei singoli oggetti selezionati nella 3D View è rappresentata dai \textit{logic bricks} (blocchi logici)
			\end{itemize}
		\end{frame}
		\begin{frame}
			\frametitle{Editor logico - logic bricks}
			\image[images/logic_bricks.png][scale=0.20]
			Funzioni programmate. Sono modificabili, combinabili e:
				\begin{itemize}
					\item sono suddivisi in tre tipi (colonne): 
					\begin{itemize}
						\item \textbf{Sensors}: si attivano al verificarsi di un evento
						\item \textbf{Controllers}: operano sul'input che ricevono dai sensori
						\item \textbf{Actuators}: interagiscono con il gioco.
					\end{itemize}
						\item sono collegati tra loro mediante \textbf{link}, che conducono il flusso logico tra \textit{sensor e controller} e \textit{controller e actuator}
					\end{itemize}
		\end{frame}
		\begin{frame}
			\frametitle{Editor logico - properties}
			\image[images/props.png][scale=0.4]
			\textbf{Properties}: variabili dei linguaggi di programmazione, usate per salvare ed accedere dati associati ad un oggetto o del gioco stesso.
		\end{frame}
		\begin{frame}
			\frametitle{Editor logico - states}
			\image[images/states.png][scale=0.3]
			\textbf{States}: stati in cui può trovarsi un oggetto, usati per definirne i diversi comportamenti. %sleeping, awake, dead
		\end{frame}
		\begin{frame}
			\frametitle{Editor di testo e console}
			\image[images/editor.png][scale=0.25]
			\image[images/console.png][scale=0.25]
		\end{frame}
	
	\section{Blocchi logici}
		\begin{frame}		%SENSORS
			\frametitle{Sensori}
			\begin{itemize}
				\item Event listeners: \textcolor{BlenderOrange}{cambiano stato} quando l'evento che attendono si verifica
				\item Di default, i sensori \textcolor{BlenderOrange}{innescano i controller} ad essi collegati ad ogni cambio di stato:
				\begin{itemize}
					\item \code{True level triggering} \\ {\footnotesize\hspace{1em} innesca i controllers solo se lo stato del sensore è positivo}
					\item \code{False level triggering} \\ {\footnotesize\hspace{1em}innesca i controllers solo se lo stato del sensore è negativo}
					\item \code{Skip} \\ {\footnotesize\hspace{1em}imposta il ritardo tra i trigger ripetuti, in frame (tick logici)}
				\end{itemize}
			\end{itemize}
		%? img
		\end{frame}		
		\begin{frame}
			\frametitle{Tipi di sensori - 1}
			I principali sensori, aggiungibili con il pulsante \code{Add Sensor}:\\
			\begin{quote}
			\textcolor{BlenderOrange}{\textbf{Nota}: Un sensore invia un impulso positivo (TRUE) a tutti i suoi controller quando si attiva; quando si disattiva invia un impulso negativo (FALSE)}
			\end{quote}
						
			\begin{itemize}
				\item \textbf{Actuator}: rileva quando un particolare actuator riceve un impulso di attivazione o di disattivazione
				\item \textbf{Always}: innesca sempre i relativi controller (ad ogni tick logico)
				\item \textbf{Collision}: rileva quando l'oggetto proprietario collide con un altro oggetto
				\item \textbf{Delay}: ritarda l'invio dell'impulso positivo
			\end{itemize}
		\end{frame}	
		\begin{frame}
			\frametitle{Tipi di Sensori - 2}
			I principali sensori, aggiungibili con il pulsante \code{Add sensor}:
			\begin{quote}
			\textcolor{BlenderOrange}{\textbf{Nota}: Un sensore invia un impulso positivo (TRUE) a tutti i suoi Controllers quando si attiva; quando si disattiva invia un impulso negativo (FALSE)}
			\end{quote}
			\begin{itemize}
				\item \textbf{Keyboard}: rileva la pressione di un tasto %Joystick--analogo
				\item \textbf{Mouse}: rileva gli eventi del mouse \\ {\footnotesize\hspace{1em}eventi: \textit{Over any, Over, Movement, Wheel down/up, R/L/M button.}}
				\item \textbf{Near}: si attiva quando rileva degli oggetti ad una distanza x dall'oggetto proprietario
				\item \textbf{Properties}: rileva cambiamenti nelle proprietà del suo oggetto
			\end{itemize}
		\end{frame}	
		
		\begin{frame}
			\frametitle{Controller}
			Blocchi logici che \textcolor{BlenderOrange}{eseguono operazioni} sull'output del sensore e, in base a determinate condizioni, \textcolor{BlenderOrange}{attivano gli actuator} ad esso connessi.
			\begin{enumerate}
				\item \textbf{States}: mostra la griglia degli stati, per spostare il controller (e tutti i blocchi ad esso collegati) in un altro state dell'oggetto
				\item Esegue il controller prima degli altri
			\end{enumerate}
			\image[images/contr.png][scale=0.3]
		\end{frame}	
		\begin{frame}
			\frametitle{Tipi di controller}
			I principali controller, aggiungibili con il pulsante \code{Add controller}:
			\begin{quote}
			\textcolor{BlenderOrange}{\textbf{Nota}: un impulso positivo (TRUE) inviato dal controller, attiva i suoi actuator; uno negativo (FALSE) li disattiva.}
			\end{quote}
			\begin{itemize}
				\item \textbf{And}: {\footnotesize TRUE se riceve \textbf{tutti} impulsi positivi, FALSE altrimenti}
				\item \textbf{Or}: {\footnotesize TRUE se riceve \textbf{almeno} un impulso positivo, FALSE altrimenti}
				\item \textbf{Nand}: {\footnotesize TRUE se riceve \textbf{almeno} un impulso negativo, FALSE se riceve \textbf{tutti} impulsi positivi}
				\item \textbf{Nor}: {\footnotesize TRUE se non riceve \textbf{nessun} impulso positivo, FALSE altrimenti}
				\item \textbf{Xor}: {\footnotesize TRUE se riceve \textbf{un solo} impulso positivo, FALSE altrimenti}
				\item \textbf{Xnor}: {\footnotesize TRUE se riceve \textbf{un solo} impulso negativo, FALSE altrimenti}
				\item \textbf{Expression}: {\footnotesize valuta un'espressione booleana scritta dall'utente}
				\item \textbf{Python}: {\footnotesize scripting in Python}
			\end{itemize}
		\end{frame}
		\begin{frame}
			\frametitle{Scripting con Python}
		E' possibile scrivere i controller in Python, il che rende possibile:
		\begin{itemize}
			\item gestire più oggetti contemporaneamente
			\item accedere ad alcune funzionalità avanzate del BGE
			\item accedere a funzionalità del Python non presenti in Blender
			\item controllare dispositivi esterni
			\item accedere a strumenti di debugging
		\end{itemize}
		E' bene sottolineare che \textcolor{BlenderOrange}{il Python non può essere utilizzato \textit{al posto} dei blocchi logici}: nel BGE, uno script va inteso come un controller più sofisticato, da utilizzare al posto di quelli booleani, e comunque qualsiasi script deve essere necessariamente chiamati da un controller.
	\end{frame}	
\begin{frame}
\frametitle{Esempio} %? da modificare
\image[images/pysample.png][scale=0.40]
Controller in Python per il controllo dell'audio.
\end{frame}
		
		\begin{frame}
			\frametitle{Actuators} %ACTUATORS
			\begin{itemize}
				\item Blocchi logici che interagiscono direttamente con il gioco
				\item Eseguono le loro funzioni solo quando ricevono un \textcolor{BlenderOrange}{impulso positivo} dai controller ad essi collegati
			\end{itemize}
		\end{frame}			
		\begin{frame}
			\frametitle{Tipi di actuator - 1}
			I principali attuatori, aggiungibili con il pulsante \code{Add actuator}:
			\begin{itemize}
				\item \textbf{Action}: {\footnotesize Riproduce e controlla le azioni di animazioni (visibile solo se è selezionata un'armatura)} %le azioni sono salvate nell'armatura!
				\item \textbf{Camera}: {\footnotesize fa sì che la camera segua un determinato oggetto}
				\item \textbf{Constraints}: {\footnotesize  pone un vincolo sull’oggetto proprietario; \\i possibili vincoli sono: Orientation Distance Location}
				\item \textbf{Edit object}: {\footnotesize permette di modificare l'oggetto proprietario; \\i possibili moidificatori sono: track to, replace mesh, end/add object}
				\item \textbf{Filter 2D}: {\footnotesize aggiunge filtri applicati al render finale degli oggetti}
				\item \textbf{Game}: {\footnotesize controlla l'intero gioco (es. ricomincia, carica, esci..)}
				\item \textbf{Mouse}: {\footnotesize Permette la visibilità del cursore e la rotazione tramite il mouse}
			\end{itemize}
		\end{frame}
				\begin{frame}
			\frametitle{Tipi di actuators - 2}
			I principali attuatori, aggiungibili con il pulsante \code{Add Actuator}:
			\begin{itemize}
				\item \textbf{Motion}: {\footnotesize controlla il movimento di un oggetto; i possibili tipi di movimento sono: Simple, Servo control, Character}
				\item \textbf{Parent}
				\item \textbf{Property}
				\item \textbf{Scene}: {\footnotesize controlla le scene nel \textit{blend-file}}
				\item \textbf{Filter 2D}: {\footnotesize aggiunge filtri applicati al render finale degli oggetti}
				\item \textbf{Steering}: {\footnotesize fornisce vari metodi di pathfinding per un determinato oggetto, ad esempio Seek, Flee, Path following}
				\item \textbf{Sound}
				\item \textbf{State}: {\footnotesize permette di gestire i vari stati del gioco}
			\end{itemize}
		\end{frame}
		
		\begin{frame}
			\frametitle{Properties}
			\begin{itemize}
				\item Equivalenti alle variabili nei linguaggi di programmazione
				\item Memorizzate nell’oggetto proprietario
				\item \textbf{Tipi di proprietà:} Timer, Float, Integer, String, Boolean
				\item Utilizzate nel debugging
			\end{itemize}
		\end{frame}	
		\begin{frame}
			\frametitle{States - 1}
			\begin{itemize}
				\item Un oggetto può possedere stati diversi, ad ogni stato corrisponde un comportamento diverso %ex expl->sleep/awake/dead
				\item Sono impostati ed usati mediante i controller: solo questi ultimi sono direttamente controllati dal sistema di stati! \image[images/states.png][scale=0.3]
				\item Ogni oggetto può trovarsi in un solo stato ad un determinato istante di tempo
			\end{itemize}
		\end{frame}	
		\begin{frame}
			\frametitle{States - 2}
			\begin{itemize}
				\item\textbf{Visible}: sono tutti gli stati disponibili. Quando uno state contiene dei logic bricks, in esso compare un pallino. Gli states visibili sono rappresentati dai quadrati di colore differente
				\item\textbf{Initial}: determina in quale tra gli states l'oggetto si troverà appena avviato il gioco
			\end{itemize}
		\end{frame}
	\section{Camera}
	\begin{frame}
	\frametitle{Caratteristiche}
		Per quanto simile alla camera nel normale sistema di rendering di Blender, è interessante soffermarsi sulla camera nel BGE, che presenta alcune caratteristiche particolari:
		\begin{itemize}
			\item Non necessariamente statica, i suoi movimenti possono essere programmati
			\item Può seguire un oggetto nel gioco (ad es. il giocatore nei giochi in prima persona)
			\item Stereo vision
			\item Dome visualization
		\end{itemize}
	\end{frame}	
	\begin{frame}
	\frametitle{Uso della camera}
	\begin{itemize}
		\item All'avvio del BGE, l'inquadratura è scelta in base all'ultima 3D view. Per utilizzare una camera in particolare, occorre selezionarla e premere \code{Numpad0} prima di avviare l'engine
		\item Per seguire un oggetto, si seleziona prima la camera, poi l'oggetto da seguire e, con \code{Ctrl+P -> Make Parent}, si rende la camera "genitore" dell'oggetto \\ {\footnotesize \hspace{1em} NB: in questo modo, la camera ruota assieme all'oggetto seguito. \\ \hspace{1em} Per evitarlo, la camera va associata a un vertice dell'oggetto}
	\end{itemize}
	\end{frame}
\end{document}
