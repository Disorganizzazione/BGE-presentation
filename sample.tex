%!TEX program = xelatex
\documentclass{beamer}

\logo{\includegraphics[height=0.8cm]{logo.png}\vspace{239pt}}


\usetheme{mhthm}

\title{Blender Game Engine}
\subtitle{Università degli Studi di Perugia}
\author{Arianna Masciolini e Giorgio Mazza}
\institute{UniPG}


\date{Maggio 2018}

\setcounter{showSlideNumbers}{1}

\begin{document}
	\setcounter{showProgressBar}{0}
	\setcounter{showSlideNumbers}{0}

	\frame{\titlepage}
	\begin{frame}
		\frametitle{Sommario}
		\begin{itemize}
			\item Introduzione \\ {\footnotesize\hspace{1em} Casi d'uso e prospettive per il Blender Game Engine}
			\item Strumenti \\ {\footnotesize\hspace{1em} Editor grafico e scripting con Python}
			\item Componenti \\ {\footnotesize\hspace{1em} Sensori, controller e attuatori}
		\end{itemize}
	\end{frame}

	\setcounter{framenumber}{0}
	\setcounter{showProgressBar}{1}
	\setcounter{showSlideNumbers}{1}
	\section{Introduzione}
		\begin{frame}
			\frametitle{Casi d'uso}
		\end{frame}
		\begin{frame}
			\frametitle{Prospettive}
		\end{frame}
	
	\section{Strumenti}
		\begin{frame}
			\frametitle{Editor grafico}
		\end{frame}
		\begin{frame}
			\frametitle{Scripting con Python}
		\end{frame}	
	
	\section{Componenti}
		\begin{frame}
			\frametitle{Sensori}
		\end{frame}
		\begin{frame}
			\frametitle{Controller}
		\end{frame}
		\begin{frame}
			\frametitle{Attuatori}
		\end{frame}

\end{document}
