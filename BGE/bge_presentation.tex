%!TEX program = xelatex
\documentclass{beamer}

\logo{\includegraphics[height=0.8cm]{logo.png}\vspace{239pt}}


\usetheme{mhthm}

\title{Blender Game Engine}
\subtitle{Università degli Studi di Perugia}
\author{Arianna Masciolini e Giorgio Mazza}
\institute{UniPG}


\date{Maggio 2018}

\setcounter{showSlideNumbers}{1}

\begin{document}
	\setcounter{showProgressBar}{0}
	\setcounter{showSlideNumbers}{0}

	\frame{\titlepage}
	\begin{frame}
		\frametitle{Sommario}
		\begin{itemize}
			\item Introduzione \\ {\footnotesize\hspace{1em} Casi d'uso e caratteristiche generali per il Blender Game Engine}
			\item Strumenti \\ {\footnotesize\hspace{1em} Editor grafico e scripting con Python}
			\item Componenti \\ {\footnotesize\hspace{1em} Sensori, controller e attuatori}
		\end{itemize}
	\end{frame}

	\setcounter{framenumber}{0}
	\setcounter{showProgressBar}{1}
	\setcounter{showSlideNumbers}{1}
	\section{Introduzione}
		\begin{frame}
			\frametitle{Caratteristiche generali}
			\begin{itemize}
				\item Casi d'uso: sviluppo di giochi e simulazioni interattive in 3D 
				\item C++
				\item Rendering in tempo reale
				\item Interfaccia: editor logico con interfaccia grafica 
				\item Supportato scripting in Python
				\item Possibilità di esportare l'applicazione per ambienti Linux, macOS, e MS-Windows
			\end{itemize}
		\end{frame}
		\begin{frame}
			\frametitle{Librerie}
			\begin{itemize}
				\item Audaspace \\ {\footnotesize\hspace{1em} Controllo dell'audio}
				\item Bullet \\ {\footnotesize\hspace{1em} Rilevamento delle collisioni e simulazione di fisica} 
				\item Recast \\ {\footnotesize\hspace{1em} Costruzione di percorsi}
				\item Detour \\ {\footnotesize\hspace{1em} Pathfinding e ragionamento spaziale}
			\end{itemize}
		\end{frame}
		\begin{frame}
			\frametitle{Prospettive}
			\textcolor{BlenderOrange}{Il BGE non sarà più supportato a partire dalla versione 2.8 di Blender}
			...
		\end{frame}
		\begin{frame}
			\frametitle{Sviluppo di applicazioni}
			\begin{enumerate}
				\item Creazione degli elementi visuali (modelli 3d o immagini)
				\item Uso dei blocchi logici per permettere l'interazione con la scena (programmazione a eventi)
				\item Creazione di una o più "macchine da presa" per il rendering e modifica dei parametri di visualizzazione
				\item Esportazione dell'applicazione
			\end{enumerate}
		\end{frame}
	
	\section{Strumenti}
		\begin{frame}
			\frametitle{Editor grafico}
			\begin{itemize}
				\item L'editor Logico è il metodo principale di impostare e modificare la logica del gioco per i vari attori che lo compongono
				\item La logica dei singoli oggetti selezionati nella 3D View è rappresentata dai Blocchi Logici (\textbf{Logic Bricks})
			\end{itemize}
		\end{frame}
		\begin{frame}
			\frametitle{Editor Grafico}
			\begin{enumerate}
			\item [img]
				\item \textbf{Logic Bricks}: funzioni programmate; Modificabili e combinabili, sono:
					\begin{itemize}
						\item Suddivisi in tre tipi (colonne): 
						\begin{itemize}
							\item\textbf{Sensors}: si attivano al verificarsi di un evento, 
							\item \textbf{Controllers}: operano sul'input che ricevono dai sensori
							\item \textbf{Actuators}: interagiscono con il gioco;
						\end{itemize}
						Quando un Sensore è positivo invia un impulso al Controllore che valuta la condizione e se è il caso innesca l’Attuatore.
						\item Collegati tra loro mediante \textbf{link}, che conducono il flusso logico tra sensor->controller e controller->actuator
					\end{itemize}
			\end{enumerate}
		\end{frame}
		\begin{frame}
			\frametitle{Editor Grafico}
			\begin{itemize}
			\item [img]
				\item[2.] \textbf{Properties}: variabili dei linguaggi di programmazione; Usate per salvare ed accedere dati associati ad un oggetto o del gioco stesso
				\item[3.] \textbf{States}: stati in cui può trovarsi un oggetto; usati per definire comportamenti diversi in situazioni diverse %sleeping, awake, dead
			\end{itemize}
		\end{frame}
		\begin{frame}		%SENSORS
			\frametitle{Logic - Sensori}
			\begin{itemize}
				\item Blocchi logici che si mettono in ascolto per un determinato evento \textit{(event listeners)} e cambiano stato quando esso si verifica.
				\item Di default, ad ogni cambio di stato, i sensori innestano i Controller ad essi collegati;
				\begin{itemize}
					\item \textit{"True level triggering"} ([img]) li innesta solo se lo stato del sensore è positivo
					\item \textit{"False level triggering"} ([img]) li innesta solo se lo stato del sensore è negativo
					\item \textit{"Freq"} imposta il ritardo tra i trigger ripetuti, in frame (tick logici)
				\end{itemize}
				\item [img]
			\end{itemize}
		\end{frame}		
		\begin{frame}
			\frametitle{Logic - Tipi di Sensori}
			Di seguito, i sensori principali, aggiungibili con il pulsante "Add Sensor":\\
			\begin{quote}
			\textbf{Nota}: Un sensore invia un impulso positivo (TRUE) a tutti i suoi Controllers quando l'evento si verifica; quando l'evento finisce invia un impulso negativo (FALSE)
			\end{quote}
						
			\begin{itemize}
				\item \textbf{Actuator}: rileva quando un particolare Actuator riceve un impulso di attivazione o di disattivazione
				\item \textbf{Always}: innesta sempre i relativi Controllers (ad ogni tick logico)
				\item \textbf{Collision}: rileva quando l'oggetto proprietario collide (tocca) un'altro oggetto
				\item \textbf{Delay}: ritarda l'invio dell'impulso positivo
			\end{itemize}
		\end{frame}	
		\begin{frame}
			\frametitle{Logic - Tipi di Sensori}
			Di seguito, i sensori principali, aggiungibili con il pulsante "Add Sensor":
			\begin{quote}
			\textbf{Nota}: Un sensore invia un impulso positivo (TRUE) a tutti i suoi Controllers quando l'evento si verifica; quando l'evento finisce invia un impulso negativo (FALSE)
			\end{quote}
			\begin{itemize}
				\item \textbf{Keyboard}: rileva la pressione di un tasto da noi assegnato %Joystick--analogo
				\item \textbf{Mouse}: rileva gli eventi del mouse. Eventi: \textit{Over any, Over, Movement, Wheel down/up, R button, L button, M button.}
				\item \textbf{Near}: si attiva quando rileva degli oggetti ad una distanza x dall'oggetto proprietario.
				\item \textbf{Properties}: rileva cambiamenti nelle Proprietà (variabili) del suo oggetto.
			\end{itemize}
		\end{frame}	
		
		\begin{frame}
			\frametitle{Logic - Controllers} %CONTROLLERS
			\begin{itemize}
				\item Blocchi logici che eseguono operazioni logiche sull'output del Sensore ed attivano gli Actuators ad esso connessi
			\end{itemize}
		\end{frame}	
		\begin{frame}
			\frametitle{Logic - Actuators} %ACTUATORS
			\begin{itemize}
				\item Blocchi logici che interagiscono direttamente con il gioco
			\end{itemize}
		\end{frame}			
		
		\begin{frame}
			\frametitle{Scripting con Python}
		\end{frame}	
		
		\begin{frame}
			\frametitle{Scripting con Python}
		\end{frame}	
	
	\section{Componenti}
		\begin{frame}
			\frametitle{Sensori}
		\end{frame}
		\begin{frame}
			\frametitle{Controller}
		\end{frame}
		\begin{frame}
			\frametitle{Attuatori}
		\end{frame}

\end{document}
