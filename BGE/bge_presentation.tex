%!TEX program = xelatex
\documentclass{beamer}

\logo{\includegraphics[height=0.8cm]{logo.png}\vspace{239pt}}


\usetheme{mhthm}

\title{Blender Game Engine}
\subtitle{Università degli Studi di Perugia}
\author{Arianna Masciolini e Giorgio Mazza}
\institute{UniPG}


\date{Maggio 2018}

\setcounter{showSlideNumbers}{1}

\begin{document}
	\setcounter{showProgressBar}{0}
	\setcounter{showSlideNumbers}{0}

	\frame{\titlepage}
	\begin{frame}
		\frametitle{Sommario}
		\begin{itemize}
			\item Introduzione \\ {\footnotesize\hspace{1em} Casi d'uso e caratteristiche generali per il Blender Game Engine}
			\item Strumenti \\ {\footnotesize\hspace{1em} Editor grafico e scripting con Python}
			\item Componenti \\ {\footnotesize\hspace{1em} Sensori, controller e attuatori}
		\end{itemize}
	\end{frame}

	\setcounter{framenumber}{0}
	\setcounter{showProgressBar}{1}
	\setcounter{showSlideNumbers}{1}
	\section{Introduzione}
		\begin{frame}
			\frametitle{Caratteristiche generali}
			\begin{itemize}
				\item Casi d'uso: sviluppo di giochi e simulazioni interattive in 3D 
				\item C++
				\item Rendering in tempo reale
				\item Interfaccia: editor logico con interfaccia grafica 
				\item Supportato scripting in Python
				\item Possibilità di esportare l'applicazione per ambienti Linux, macOS, e MS-Windows
			\end{itemize}
		\end{frame}
		\begin{frame}
			\frametitle{Librerie}
			\begin{itemize}
				\item Audaspace \\ {\footnotesize\hspace{1em} Controllo dell'audio}
				\item Bullet \\ {\footnotesize\hspace{1em} Rilevamento delle collisioni e simulazione di fisica} 
				\item Recast \\ {\footnotesize\hspace{1em} Costruzione di percorsi}
				\item Detour \\ {\footnotesize\hspace{1em} Pathfinding e ragionamento spaziale}
			\end{itemize}
		\end{frame}
		\begin{frame}
			\frametitle{Prospettive}
			\begin{quotation}\small
				"(...) I also propose to refocus the current game engine (...) I propose to make the GE to become a real part of Blender code – to make it not separated anymore. This would make it more supported, more stable and (I”m sure) much more fun to work on as well. (...) What should then be dropped is the idea to make Blender have an embedded “true” game engine. (...) We never managed to make something with the portability and quality of Unreal or Crysis… or even Unity3D. And Blender's GPL license is not helping here much either. (...) The main cool feature of our GE is that it was integrated with a 3D tool, to allow people to make 3D interaction for walkthroughs, for scientific sims, or game prototypes. If we bring back this (original) design focus for a GE, I think we still get something unique and cool, with seamless integration of realtime and “offline” 3D".
				\\ \textcolor{BlenderOrange}{\textbf{-Ton Roosendaal, creator of Blender}}
			\end{quotation}
		\end{frame}
		\begin{frame}
			\frametitle{Fasi dello sviluppo}
			\begin{enumerate}
				\item Creazione degli elementi visuali (modelli 3d o immagini)
				\item Uso dei blocchi logici per permettere l'interazione con la scena (programmazione a eventi)
				\item Creazione di una o più "macchine da presa" per il rendering e modifica dei parametri di visualizzazione
				\item Esportazione dell'applicazione
			\end{enumerate}
		\end{frame}
	
	\section{Strumenti}
		\begin{frame}
			\frametitle{Editor grafico}
		\end{frame}
		\begin{frame}
			\frametitle{Scripting con Python}
		\end{frame}	
	
	\section{Componenti}
		\begin{frame}
			\frametitle{Sensori}
		\end{frame}
		\begin{frame}
			\frametitle{Controller}
		\end{frame}
		\begin{frame}
			\frametitle{Attuatori}
		\end{frame}

\end{document}
