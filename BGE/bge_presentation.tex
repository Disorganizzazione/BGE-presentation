%!TEX program = xelatex
\documentclass{beamer}

\logo{\includegraphics[height=0.8cm]{logo.png}\vspace{239pt}}


\usetheme{mhthm}

\title{Blender Game Engine}
\subtitle{Università degli Studi di Perugia}
\author{Arianna Masciolini e Giorgio Mazza}
\institute{UniPG}


\date{Maggio 2018}

\setcounter{showSlideNumbers}{1}

\begin{document}
	\setcounter{showProgressBar}{0}
	\setcounter{showSlideNumbers}{0}

	\frame{\titlepage}
	\begin{frame}
		\frametitle{Sommario}
		\begin{itemize}
			\item Introduzione \\ {\footnotesize\hspace{1em} Casi d'uso e caratteristiche generali per il Blender Game Engine}
			\item Strumenti \\ {\footnotesize\hspace{1em} Editor grafico e scripting con Python}
			\item Componenti \\ {\footnotesize\hspace{1em} Sensori, controller e attuatori}
		\end{itemize}
	\end{frame}

	\setcounter{framenumber}{0}
	\setcounter{showProgressBar}{1}
	\setcounter{showSlideNumbers}{1}
	\section{Introduzione}
		\begin{frame}
			\frametitle{Caratteristiche generali}
			\begin{itemize}
				\item Casi d'uso: sviluppo di giochi e simulazioni interattive in 3D 
				\item C++
				\item Rendering in tempo reale
				\item Interfaccia: editor logico con interfaccia grafica 
				\item Supportato scripting in Python
				\item Possibilità di esportare l'applicazione per ambienti Linux, macOS, e MS-Windows
			\end{itemize}
		\end{frame}
		\begin{frame}
			\frametitle{Librerie}
			\begin{itemize}
				\item Audaspace \\ {\footnotesize\hspace{1em} Controllo dell'audio}
				\item Bullet \\ {\footnotesize\hspace{1em} Rilevamento delle collisioni e simulazione di fisica} 
				\item Recast \\ {\footnotesize\hspace{1em} Costruzione di percorsi}
				\item Detour \\ {\footnotesize\hspace{1em} Pathfinding e ragionamento spaziale}
			\end{itemize}
		\end{frame}
		\begin{frame}
			\frametitle{Prospettive}
			\textcolor{BlenderOrange}{Il BGE non sarà più supportato a partire dalla versione 2.8 di Blender}
			...
		\end{frame}
		\begin{frame}
			\frametitle{Sviluppo di applicazioni}
			\begin{enumerate}
				\item Creazione degli elementi visuali (modelli 3d o immagini)
				\item Uso dei blocchi logici per permettere l'interazione con la scena (programmazione a eventi)
				\item Creazione di una o più "macchine da presa" per il rendering e modifica dei parametri di visualizzazione
				\item Esportazione dell'applicazione
			\end{enumerate}
		\end{frame}
	
	\section{Strumenti}
		\begin{frame}
			\frametitle{Editor grafico}
		\end{frame}
		\begin{frame}
			\frametitle{Scripting con Python}
		\end{frame}	
	
	\section{Componenti}
		\begin{frame}
			\frametitle{Sensori}
		\end{frame}
		\begin{frame}
			\frametitle{Controller}
		\end{frame}
		\begin{frame}
			\frametitle{Attuatori}
		\end{frame}

\end{document}
